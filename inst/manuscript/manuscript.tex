% Options for packages loaded elsewhere
\PassOptionsToPackage{unicode}{hyperref}
\PassOptionsToPackage{hyphens}{url}
\PassOptionsToPackage{dvipsnames,svgnames,x11names}{xcolor}
%
\documentclass[
  article]{jss}

\usepackage{amsmath,amssymb}
\usepackage{iftex}
\ifPDFTeX
  \usepackage[T1]{fontenc}
  \usepackage[utf8]{inputenc}
  \usepackage{textcomp} % provide euro and other symbols
\else % if luatex or xetex
  \usepackage{unicode-math}
  \defaultfontfeatures{Scale=MatchLowercase}
  \defaultfontfeatures[\rmfamily]{Ligatures=TeX,Scale=1}
\fi
\usepackage{lmodern}
\ifPDFTeX\else
    % xetex/luatex font selection
\fi
% Use upquote if available, for straight quotes in verbatim environments
\IfFileExists{upquote.sty}{\usepackage{upquote}}{}
\IfFileExists{microtype.sty}{% use microtype if available
  \usepackage[]{microtype}
  \UseMicrotypeSet[protrusion]{basicmath} % disable protrusion for tt fonts
}{}
\makeatletter
\@ifundefined{KOMAClassName}{% if non-KOMA class
  \IfFileExists{parskip.sty}{%
    \usepackage{parskip}
  }{% else
    \setlength{\parindent}{0pt}
    \setlength{\parskip}{6pt plus 2pt minus 1pt}}
}{% if KOMA class
  \KOMAoptions{parskip=half}}
\makeatother
\usepackage{xcolor}
\setlength{\emergencystretch}{3em} % prevent overfull lines
\setcounter{secnumdepth}{-\maxdimen} % remove section numbering
% Make \paragraph and \subparagraph free-standing
\ifx\paragraph\undefined\else
  \let\oldparagraph\paragraph
  \renewcommand{\paragraph}[1]{\oldparagraph{#1}\mbox{}}
\fi
\ifx\subparagraph\undefined\else
  \let\oldsubparagraph\subparagraph
  \renewcommand{\subparagraph}[1]{\oldsubparagraph{#1}\mbox{}}
\fi


\providecommand{\tightlist}{%
  \setlength{\itemsep}{0pt}\setlength{\parskip}{0pt}}\usepackage{longtable,booktabs,array}
\usepackage{calc} % for calculating minipage widths
% Correct order of tables after \paragraph or \subparagraph
\usepackage{etoolbox}
\makeatletter
\patchcmd\longtable{\par}{\if@noskipsec\mbox{}\fi\par}{}{}
\makeatother
% Allow footnotes in longtable head/foot
\IfFileExists{footnotehyper.sty}{\usepackage{footnotehyper}}{\usepackage{footnote}}
\makesavenoteenv{longtable}
\usepackage{graphicx}
\makeatletter
\def\maxwidth{\ifdim\Gin@nat@width>\linewidth\linewidth\else\Gin@nat@width\fi}
\def\maxheight{\ifdim\Gin@nat@height>\textheight\textheight\else\Gin@nat@height\fi}
\makeatother
% Scale images if necessary, so that they will not overflow the page
% margins by default, and it is still possible to overwrite the defaults
% using explicit options in \includegraphics[width, height, ...]{}
\setkeys{Gin}{width=\maxwidth,height=\maxheight,keepaspectratio}
% Set default figure placement to htbp
\makeatletter
\def\fps@figure{htbp}
\makeatother

\usepackage{orcidlink,thumbpdf,lmodern}

\newcommand{\class}[1]{`\code{#1}'}
\newcommand{\fct}[1]{\code{#1()}}
\makeatletter
\@ifpackageloaded{caption}{}{\usepackage{caption}}
\AtBeginDocument{%
\ifdefined\contentsname
  \renewcommand*\contentsname{Table of contents}
\else
  \newcommand\contentsname{Table of contents}
\fi
\ifdefined\listfigurename
  \renewcommand*\listfigurename{List of Figures}
\else
  \newcommand\listfigurename{List of Figures}
\fi
\ifdefined\listtablename
  \renewcommand*\listtablename{List of Tables}
\else
  \newcommand\listtablename{List of Tables}
\fi
\ifdefined\figurename
  \renewcommand*\figurename{Figure}
\else
  \newcommand\figurename{Figure}
\fi
\ifdefined\tablename
  \renewcommand*\tablename{Table}
\else
  \newcommand\tablename{Table}
\fi
}
\@ifpackageloaded{float}{}{\usepackage{float}}
\floatstyle{ruled}
\@ifundefined{c@chapter}{\newfloat{codelisting}{h}{lop}}{\newfloat{codelisting}{h}{lop}[chapter]}
\floatname{codelisting}{Listing}
\newcommand*\listoflistings{\listof{codelisting}{List of Listings}}
\makeatother
\makeatletter
\makeatother
\makeatletter
\@ifpackageloaded{caption}{}{\usepackage{caption}}
\@ifpackageloaded{subcaption}{}{\usepackage{subcaption}}
\makeatother
\makeatletter
\@ifpackageloaded{tcolorbox}{}{\usepackage[skins,breakable]{tcolorbox}}
\makeatother
\makeatletter
\@ifundefined{shadecolor}{\definecolor{shadecolor}{rgb}{.97, .97, .97}}{}
\makeatother
\makeatletter
\makeatother
\makeatletter
\ifdefined\Shaded\renewenvironment{Shaded}{\begin{tcolorbox}[boxrule=0pt, enhanced, sharp corners, borderline west={3pt}{0pt}{shadecolor}, frame hidden, interior hidden, breakable]}{\end{tcolorbox}}\fi
\makeatother
\ifLuaTeX
  \usepackage{selnolig}  % disable illegal ligatures
\fi
\usepackage{bookmark}

\IfFileExists{xurl.sty}{\usepackage{xurl}}{} % add URL line breaks if available
\urlstyle{same} % disable monospaced font for URLs
\hypersetup{
  pdftitle={Spatial Generalized Linear Models in  Using },
  pdfauthor={Michael Dumelle; Jay M. Ver Hoef; Matt Higham},
  pdfkeywords={Autoregressive Model, Geostatistics, Spatial
Covariance, Spatially-Explicit Model, Statistical Model},
  colorlinks=true,
  linkcolor={blue},
  filecolor={Maroon},
  citecolor={Blue},
  urlcolor={Blue},
  pdfcreator={LaTeX via pandoc}}

%% -- Article metainformation (author, title, ...) -----------------------------

%% Author information
\author{Michael Dumelle~\orcidlink{0000-0002-3393-5529}\\United States\\
Environmental Protection Agency \And Jay M. Ver
Hoef~\orcidlink{0000-0003-4302-6895}\\National Oceanic and\\
Atmospheric Administration \AND Matt
Higham~\orcidlink{0009-0006-4217-625X}\\St.~Lawrence University}
\Plainauthor{Michael Dumelle, Jay M. Ver Hoef, Matt
Higham} %% comma-separated

\title{Spatial Generalized Linear Models in \proglang{R} Using
\pkg{spmodel}}
\Plaintitle{Spatial Generalized Linear Models in
Using} %% without formatting

%% an abstract and keywords
\Abstract{Generalized linear models (GLMs) describe a non-normal
response variable that may be binary, count, skewed, or a proportion.
Typically, observations in a GLM are assumed independent of one another.
For spatial data, this independence assumption is impractical, as nearby
locations tend to be more similar than locations far apart. The
\pkg{spmodel} \proglang{R} package provides tools to fit GLMs that
incorporate spatial autocorrelation (i.e., spatial generalized linear
models, or SPGLMs). SPGLMs are fit in \pkg{spmodel} using a novel
application of the Laplace approximation via \fct{spglm} for
point-referenced data or \fct{spgautor} for areal (i.e., lattice), data.
\fct{spglm} and \fct{spgautor} closely resemble \fct{glm} from base
\proglang{R} but include arguments that control the spatial
autocorrelation structure. \pkg{spmodel} has many helper functions for
model inspection and diagnostics, some of which leverage other
\proglang{R} packages like \pkg{broom} and \pkg{emmeans}. \pkg{spmodel}
has tools to make predictions of the latent spatial-mean process at
unobserved locations. \pkg{spmodel} also provides many advanced features
like accommodating geometric anisotropy and nonspatial random effects,
simulating spatially autocorrelated data, and more. Here we use
\pkg{spmodel} to illustrate the modeling of binary, count, skewed and
proportion response variables from several point-referenced and areal
data sets.}

%% at least one keyword must be supplied
\Keywords{Autoregressive Model, Geostatistics, Spatial
Covariance, Spatially-Explicit Model, Statistical Model}
\Plainkeywords{Autoregressive Model, Geostatistics, Spatial
Covariance, Spatially-Explicit Model, Statistical Model}

%% publication information
%% NOTE: Typically, this can be left commented and will be filled out by the technical editor
%% \Volume{50}
%% \Issue{9}
%% \Month{June}
%% \Year{2012}
%% \Submitdate{2012-06-04}
%% \Acceptdate{2012-06-04}
%% \setcounter{page}{1}
%% \Pages{1--xx}

%% The address of (at least) one author should be given
%% in the following format:
\Address{
Michael Dumelle\\
Pacific Ecological Systems Division\\
200 SW 35th St\\
Corvallis OR\\
E-mail: \email{dumelle.michael@epa.gov}\\
\\~
Jay M. Ver Hoef\\
Alaska Fisheries Science Center\\
\\~
Matt Higham\\
Department of Math, Computer Science, and Statistics\\
\\~

}

\begin{document}
\maketitle

\newpage

\section{Introduction}\label{sec-intro}

Binary, count, proportion, and skewed data are ubiquitous in practice.
These data types are naturally modeled using a generalized linear model
(GLM) framework
\citep{nelder1972generalized, mccullagh1989generalized, myers2012generalized, faraway2016extending}.
In a GLM, a response variable \(\text{y}\) belongs to a particular
statistical distribution with mean \(\mu\). For example, \(\text{y}\)
may be distributed as a Poisson random variable with some mean \(\mu\).
Based on the response distribution assumed for \(\text{y}\), GLMs link a
function of \(\mu\) to explanatory variables via a link function:
\begin{equation}\label{eq-glm}
f(\boldsymbol{\mu}) \equiv \mathbf{w} = \mathbf{X} \boldsymbol{\beta},
\end{equation} where for a sample size \(n\), \(\boldsymbol{\mu}\) is
the mean of an \(n \times 1\) response vector \(\mathbf{y}\),
\(f(\boldsymbol{\mu})\) is the link function that connects
\(\boldsymbol{\mu}\) to \(\mathbf{w}\), \(\mathbf{X}\) is the
\(n \times p\) design matrix of explanatory variables, and
\(\boldsymbol{\beta}\) is the \(p \times 1\) vector of fixed effects.
The mean, \(\boldsymbol{\mu}\), is usually constrained in some way
(e.g., positive) that depends on the distribution assumed for
\(\mathbf{y}\), but \(\mathbf{w}\) is unconstrained. The parameters in
Equation\(~\)\eqref{eq-glm} are typically estimated via maximum
likelihood (e.g., iteratively reweighted least squares)
\citep{chambers1992S}. The \fct{glm} function is commonly used to fit
GLMs in the \proglang{R} programming language \citep{rcore2024}.

The GLM framework in Equation\(~\)\eqref{eq-glm} assumes the elements of
\(\mathbf{y}\) are independent of one another. This assumption is
impractical for spatial data, where nearby observations tend to be more
similar than distant observations \citep{tobler1970computer}. Ignoring
this spatial dependence can give rise to misleading inference and poor
prediction \citep{zimmerman2024spatial}. Spatial GLMs (SPGLMs) formally
incorporate spatial autocorrelation into a GLM by adding to
Equation\(~\)\eqref{eq-glm} two random effects that elucidate spatial
structure: \begin{equation}\label{eq-spglm}
f(\boldsymbol{\mu}) \equiv \mathbf{w} = \mathbf{X} \boldsymbol{\beta} + \boldsymbol{\tau} + \boldsymbol{\epsilon},
\end{equation} where \(\boldsymbol{\tau}\) is an \(n \times 1\) column
vector of spatially dependent random errors, and
\(\boldsymbol{\epsilon}\) is an \(n \times 1\) column vector of
spatially independent random errors. We make a few assumptions about
\(\boldsymbol{\tau}\) and \(\boldsymbol{\epsilon}\): first, that
\(\text{E}(\boldsymbol{\tau}) = \text{E}(\boldsymbol{\epsilon}) = \boldsymbol{0}\),
where \(\text{E}(\cdot)\) denotes expectation; second, that
\(\text{Cov}(\boldsymbol{\tau}) = \sigma^2_\tau \mathbf{R}\), where
\(\mathbf{R}\) is an \(n \times n\) matrix that determines the spatial
dependence structure in \(\mathbf{w}\) (which is linked to
\(\boldsymbol{\mu}\) and hence \(\mathbf{y}\)) and depends on a range
parameter, \(\phi\); third, that
\(\text{Cov}(\boldsymbol{\epsilon}) = \sigma^2_\epsilon \mathbf{I}\),
where \(\mathbf{I}\) is an \(n \times n\) identity matrix; and fourth,
that \(\boldsymbol{\tau}\) and \(\boldsymbol{\epsilon}\) are independent
of one another. The parameter \(\sigma^2_{\tau}\) is called the
spatially dependent random error variance or partial sill. The parameter
\(\sigma^2_\epsilon\) is called the spatially independent random error
variance or nugget. These two variance parameters are henceforth more
intuitively written as \(\sigma^2_{de}\) and \(\sigma^2_{ie}\),
respectively. The covariance of \(\mathbf{w}\) is denoted
\(\boldsymbol{\Sigma}\) and is given by \begin{equation}\label{eq-spcov}
 \boldsymbol{\Sigma} = \sigma^2_{de}\mathbf{R} + \sigma^2_{ie} \mathbf{I}.
\end{equation} The parameters \(\sigma^2_{de}\), \(\phi\), and
\(\sigma^2_{ie}\) are elements of \(\boldsymbol{\theta}\), the
covariance parameter vector.

The \pkg{spmodel} \proglang{R} package provides tools for fitting
spatial statistical models and making predictions at unobserved
locations \citep{dumelle2023spmodel}. A recent major update (v0.4.0) to
\pkg{spmodel} added SPGLM (Equation\(~\)\ref{eq-spglm}) support for
binary, count, skewed, and proportion response variables
(Table\(~\)\ref{table-links}), greatly expanding the class of models
that \pkg{spmodel} makes accessible to practitioners.

\begin{table}

\centering{

\centering
\begin{tabular}{||c|ccc||}
 \hline
 Family & Link Function & Link Name & Data Type  \\ [0.5ex]
 \hline \hline
 Binomial & $f(\mu) = \log(\mu / (1 - \mu))$ & Logit & Binary; Binary Count \\
 Poisson & $f(\mu) = \log(\mu)$ & Log & Count \\
 Negative Binomial & $f(\mu) = \log(\mu)$ & Log & Count \\
 Beta & $f(\mu) = \log(\mu / (1 - \mu))$ & Logit & Proportion \\
 Gamma & $f(\mu) = \log(\mu)$ & Log & Skewed \\
 Inverse Gaussian & $f(\mu) = \log(\mu)$ & Log & Skewed \\
 \hline\hline
\end{tabular}

}

\caption{\label{table-links}SPGLM response distributions and their link functions and data types.}

\end{table}%

SPGLMs for point-referenced data are called geostatistical GLMs and are
fit using \fct{spglm}. Data are point-referenced when the elements in
\(\mathbf{y}\) are observed at point-locations indexed by x-coordinates
and y-coordinates on a spatially continuous surface with an infinite
number of locations (e.g., point locations in a field). SPGLMs for areal
data are called spatial autoregresive models and are fit using
\fct{spgautor}. Data are areal (i.e., lattice) when they are part of a
finite network of polygons whose connections are indexed by a
neighborhood structure (e.g., states in a country).

Several other \proglang{R} packages exist for analyzing SPGLMs. The
\pkg{brms} \citep{burkner2017brms}, \pkg{carBayes}
\citep{lee2013carbayes}, \pkg{ngspatial} \citep{hughes2020ngspatial},
\pkg{R-INLA} \citep{lindgren2015bayesian}, \pkg{spBayes}
\citep{finley2007spbayes}, and \pkg{spNNGP} \citep{finley2002spnngp}
packages take a Bayesian approach, either directly sampling from
posterior distributions of parameters (e.g., using MCMC) or
approximating them. A benefit of Bayesian approaches is that prior
information can be incorporated and uncertainty quantification of
parameter estimates is straightforward. However, Bayesian approaches,
especially those using MCMC, tend to be computationally expensive. In
order to reduce computation time, many of these packages work with the
precision matrix instead of the covariance matrix so that inverting
matrices (a computational bottleneck) is not required. For example,
\pkg{R-INLA} uses the precision matrix (often directly modeled as with
autoregressive models) and tends to be very fast. Working with precision
matrices, however, can be more restrictive and less intuitive than
working directly with the covariance matrix. The \pkg{FRK}
\citep{sainsbury2024modeling}, \pkg{glmmTMB} \citep{brooks2017glmmtmb},
\pkg{hglm} \citep{ronnegard2010hglm}, \pkg{mgcv}
\citep{wood2017generalized}, and \pkg{spaMM} \citep{rousset2014spamm}
packages directly use Laplace, quasi-likelihood, or reduced-rank
approaches to estimate parameters. These direct approaches tend to be
computationally efficient, as they don't rely on MCMC sampling. In
contrast to the Bayesian approach, a drawback of these direct approaches
is that prior information cannot be formally incorporated and covariance
parameter uncertainty is challenging to quantify. SPGLMs in
\pkg{spmodel} are most similar to those in \pkg{glmmTMB} --
\pkg{spmodel} uses analytical solutions for maximizing the likelihood
(which we describe in the next section), while \pkg{glmmTMB} uses
automatic differentiation for maximizing the likelihood (which tends to
be much slower than analytical solutions).

Missing from the aforementioned \proglang{R} packages is the complete
set of tools for SPGLMs that \pkg{spmodel} provides. Importantly, the
\fct{spglm} and \fct{spautor} functions act as a spatial analogue to the
familiar \fct{glm} from base \proglang{R}, making the transition from
GLMs to SPGLMs relatively seamless. \pkg{spmodel} leverages many
commonly used \proglang{R} generics like \fct{summary} to better
understand fitted models. Six GLM families (Table\(~\)\ref{table-links})
and 20 spatial covariance (or precision) functions are supported. Also
available are functions for data visualization, model fitting, model
summaries, model diagnostics, model comparison, and prediction, all
crucial components of a data analysis. Through extra function arguments,
\pkg{spmodel} supports advanced features like geometric anisotropy,
non-spatial random effects, methods for large data sets, and more.
Importantly, \pkg{spmodel} extends popular packages like \pkg{broom}
\citep{robinson2021broom, kuhn2022tidy} and \pkg{emmeans}
\citep{lenth2024emmeans}. \pkg{spmodel} also provides function for
simulating SPGLM data (e.g., \fct{sprbinom}, \fct{sprpois}).

The rest of this article is organized as follows. In
Section~\ref{sec-spglm}, we provide some background for the SPGLM
fitting and prediction routines in \pkg{spmodel}. In
Section~\ref{sec-applications}, we provide several applications of
\pkg{spmodel} to spatial binary, count, skewed and proportion data with
both point-referenced and areal supports. And in
Section~\ref{sec-discussion}, we end with a discussion synthesizing
\pkg{spmodel}'s contributions to the analysis of SPGLMs in \proglang{R}.

\section{Spatial generalized linear models using the Laplace
approximation}\label{sec-spglm}

GLMs with random effects are often written hierarchically
\citep{lee1996hierarchical, bolker2009generalized, wood2017generalized}.
\pkg{spmodel} leverages this hierarchical structure and uses a novel
application of the Laplace approximation \citep{ver2024marginal} to fit
models. This marginal approach is quite flexible, accommodating a wide
range of possible dependence structures and formally maximizing a
likelihood. Maximizing a likelihood yields convenient likelihood-based
statistics like AIC \citep{akaike1974new}, AICc
\citep{hoeting2006model}, BIC \citep{schwarz1978estimating}, deviance
\citep{mccullagh1989generalized}, and likelihood ratio tests for model
comparison, a benefit compared to quasi-likelihood
\citep{wedderburn1974quasi, breslow1993approximate} or pseudo-likelihood
approaches \citep{wolfinger1993generalized}, which only specify the
first two moments of a distribution. \citet{ver2024marginal} provide
thorough context for the marginal approach and its associated details,
but next we provide an short overview of the methodology.

Our goal is to marginalize over the latent mean \(\mathbf{w}\) and fixed
effects \(\boldsymbol{\beta}\) in Equation\(~\)\eqref{eq-spglm} to
obtain a response distribution for \(\mathbf{y}\) that depends only on
the explanatory variables, \(\mathbf{X}\), the response distribution's
dispersion parameter, \(\varphi\), and the covariance parameters,
\(\boldsymbol{\theta}\). We can represent this marginal distribution
hierarchically as \begin{equation}\label{eq-marginal}
  [\mathbf{y}|\mathbf{X}, \varphi, \boldsymbol{\theta}] = \int_{\mathbf{w}} \int_{\boldsymbol{\beta}} [\mathbf{y} | f^{-1}(\mathbf{w}), \varphi] [\mathbf{w} | \mathbf{X}, \boldsymbol{\theta}] d\boldsymbol{\beta} d\mathbf{w} .
\end{equation} In Equation\(~\)\eqref{eq-marginal},
\([\mathbf{y} | f^{-1}(\mathbf{w}), \varphi]\) is the density for
\(\mathbf{y}\) (e.g., binomial, Poisson) given the latent mean and
dispersion parameter and
\([\mathbf{w} | \mathbf{X}, \boldsymbol{\theta}]\) is the Gaussian
density for \(\mathbf{w}\) given the explanatory variables, fixed
effects, and covariance parameters. Integrating \(\boldsymbol{\beta}\)
out of Equation\(~\)\eqref{eq-spglm} yields
\begin{equation}\label{eq-marginal2}
  [\mathbf{y}|\mathbf{X}, \varphi, \boldsymbol{\theta}] = \int_{\mathbf{w}} [\mathbf{y} | f^{-1}(\mathbf{w}), \varphi] [\mathbf{w} | \mathbf{X}, \boldsymbol{\theta}] d\mathbf{w},
\end{equation} where \([\mathbf{w} | \mathbf{X}, \boldsymbol{\theta}]\)
is the restricted (i.e., residual) Gaussian density
\citep[\citet{harville1977maximum},
\citet{wolfinger1994computing}]{patterson1971recovery} for
\(\mathbf{w}\) given the explanatory variables and covariance
parameters. This restricted Gaussian density is given by
\begin{equation}\label{eq-reml-def}
[\mathbf{w} | \mathbf{X}, \boldsymbol{\theta}] = \frac{\exp(-\frac{1}{2}(\mathbf{y} - \mathbf{X}\tilde{\boldsymbol{\beta}}) \boldsymbol{\Sigma}^{-1} (\mathbf{y} - \mathbf{X}\tilde{\boldsymbol{\beta}})^T)}{(2 \pi)^{(n - p)/2} |\boldsymbol{\Sigma}|^{1/2}|\mathbf{X}^T \boldsymbol{\Sigma}^{-1} \mathbf{X}|^{1/2}},
\end{equation} where
\(\tilde{\boldsymbol{\beta}} = (\mathbf{X}^T \boldsymbol{\Sigma}^{-1} \mathbf{X})^{-1} \mathbf{X}^T \boldsymbol{\Sigma}^{-1} \mathbf{w}\)
and \(|\cdot|\) denotes the determinant.

Next, let
\(\ell_\mathbf{w} = \log([\mathbf{y} | f^{-1}(\mathbf{w}), \varphi] [\mathbf{w} | \mathbf{X}, \boldsymbol{\theta}])\)
and then notice that Equation\(~\)\eqref{eq-marginal2} can be written as
\begin{equation}\label{eq-marginal3}
  [\mathbf{y}|\mathbf{X}, \varphi, \boldsymbol{\theta}] = \int_{\mathbf{w}} \exp(\ell_\mathbf{w}) d\mathbf{w}.
\end{equation} A Taylor series expansion of \(\ell_\mathbf{w}\) around a
point \(\mathbf{a}\) yields \begin{equation}\label{eq-marginal4}
  [\mathbf{y}|\mathbf{X}, \varphi, \boldsymbol{\theta}] \approx \int_{\mathbf{w}} \exp(\ell_\mathbf{a} + \mathbf{g}^T(\mathbf{w} - \mathbf{a}) + \frac{1}{2}(\mathbf{w} - \mathbf{a})^T \mathbf{G} (\mathbf{w} - \mathbf{a}))d\mathbf{w},
\end{equation} where \(\mathbf{g}\) and \(\mathbf{G}\) are the gradient
and Hessian, respectively, of \(\ell_\mathbf{w}\) with respect to
\(\mathbf{w}\). If \(\mathbf{a}\) is a value for which
\(\mathbf{g} = \mathbf{0}\), \begin{equation}\label{eq-marginal5}
  [\mathbf{y}|\mathbf{X}, \varphi, \boldsymbol{\theta}] \approx \exp(\ell_\mathbf{a}) \int_{\mathbf{w}} \exp(-\frac{1}{2}(\mathbf{w} - \mathbf{a})^T (-\mathbf{G}) (\mathbf{w} - \mathbf{a}))d\mathbf{w}.
\end{equation} Notice that the integral in
Equation\(~\)\ref{eq-marginal5} can be solved in the same manner as the
normalizing constant in a multivariate Gaussian distribution, and
rewriting \(\exp(\ell_\mathbf{a})\) yields
\begin{equation}\label{eq-marginal6}
  [\mathbf{y}|\mathbf{X}, \varphi, \boldsymbol{\theta}] \approx [\mathbf{y} | f^{-1}(\mathbf{a}), \varphi] [\mathbf{a} | \mathbf{X}, \boldsymbol{\theta}] (2 \pi)^{n/2}|-\mathbf{G}_{\mathbf{a}}|^{-1/2}.
\end{equation} \citet{ver2024marginal} show how to evaluate
\(\mathbf{g}\) and \(\mathbf{G}\) to obtain \textbf{a} and maximize (the
natural logarithm of) Equation\(~\)\eqref{eq-marginal6} for the six
response distributions in Table\(~\)\ref{table-links}. Maximizing
Equation\(~\)\eqref{eq-marginal6} using a Newton-Rhapson approach yields
the marginal restricted maximum likelihood estimators \(\hat{\varphi}\)
and \(\hat{\boldsymbol{\theta}}\). \citet{ver2024marginal} show that the
fixed effect estimator is
\(\hat{\boldsymbol{\beta}} = (\mathbf{X}^T \hat{\boldsymbol{\Sigma}}^{-1}\mathbf{X})^{-1}\mathbf{X}^T \hat{\boldsymbol{\Sigma}}^{-1}\mathbf{w}\),
where \(\hat{\boldsymbol{\Sigma}}\) is the covariance matrix
\(\boldsymbol{\Sigma}\) evaluated at \(\hat{\boldsymbol{\theta}}\).
\citet{ver2024marginal} also derive the covariance matrix of
\(\hat{\boldsymbol{\beta}}\) and show how to predict \(\mathbf{w}\) at
unobserved locations and quantify their uncertainties.

\section{Application}\label{sec-applications}

The \fct{spglm} (for point-referenced data) and \fct{spgautor} (for
areal data) functions in \pkg{spmodel} fit SPGLMs using the Laplace
approximation outlined in Section~\ref{sec-spglm}. Both \fct{spglm} and
\fct{spgautor} generally require the following four arguments:
\texttt{formula}, a formula that describes the relationship between the
response variable and explanatory variables; \texttt{family}, the
response distribution (which can be \texttt{binomial}, \texttt{poisson},
\texttt{nbinomial}, \texttt{Gamma}, \texttt{inverse.gaussian}, or
\texttt{beta}); \texttt{data}, the data frame that holds the variables
in \texttt{formula} as well as spatial locations; and
\texttt{spcov\_type}, the spatial covariance type. The first three
arguments are shared by \fct{glm}; thus, the transition from GLMs to
SPGLMs requires only one additional argument: \texttt{spcov\_type}. The
\fct{spglm} spatial covariance types measure dependence as a function of
Euclidean distance among observations; an example is the exponential
spatial covariance: \begin{equation}\label{eq-spcov-exp}
  \boldsymbol{\Sigma} = \sigma^2_{de} \exp(-\mathbf{H}/\phi) + \sigma^2_{ie}\mathbf{I},
\end{equation} where \(\mathbf{H}\) is a matrix of pairwise distances
among all observations. \fct{spglm} currently supports 18 distinct
spatial covariance functions.

The \fct{spgautor} spatial covariance types measure dependence as a
function of neighborhood distance among observations; an example is the
simultaneous autoregressive covariance matrix:
\begin{equation}\label{eq-spcov-sar}
  \boldsymbol{\Sigma} = \sigma^2_{de} [(\mathbf{I} - \phi \mathbf{W})(\mathbf{I} - \phi \mathbf{W})^T]^{-1} + \sigma^2_{ie}\mathbf{I},
\end{equation} where \(\mathbf{W}\) is a matrix that represents the
neighborhood structure among all observations. \fct{spgautor} currently
supports two distinct spatial covariance functions.

In the rest of this section, we use \pkg{spmodel} to study binary,
count, skewed, and proportion response variables that are either
point-referenced or areal. We use \pkg{spmodel} for all parts of the
data analysis, from estimation to inference to model diagnostics to
prediction. We first describe core \pkg{spmodel} functionality in an
application to binary data, while additional analyses highlight count,
skewed, and proportion data as well as some additional \pkg{spmodel}
features. Before proceeding, load \pkg{spmodel} into the current
\proglang{R} session:

\begin{verbatim}
R> library("spmodel")
\end{verbatim}

\subsection{Binary data}\label{binary-data}

The \texttt{moose} data in \pkg{spmodel} contain information on moose
presence in Alaska. They are an \texttt{sf} object, a special data frame
that is supplemented with spatial information using the \pkg{sf} package
\citep{pebesma2018sf}. The first few rows look like:

\begin{verbatim}
R> head(moose)
\end{verbatim}

\begin{verbatim}
Simple feature collection with 6 features and 4 fields
Geometry type: POINT
Dimension:     XY
Bounding box:  xmin: 281896.4 ymin: 1518398 xmax: 311325.3 ymax: 1541016
Projected CRS: NAD83 / Alaska Albers
# A tibble: 6 x 5
   elev strat count presence           geometry
  <dbl> <chr> <dbl> <fct>           <POINT [m]>
1  469. L         0 0        (293542.6 1541016)
2  362. L         0 0        (298313.1 1533972)
3  173. M         0 0        (281896.4 1532516)
4  280. L         0 0        (298651.3 1530264)
5  620. L         0 0        (311325.3 1527705)
6  164. M         0 0        (291421.5 1518398)
\end{verbatim}

There are five columns: \texttt{elev}, the numeric site elevation
(meters); \texttt{strat} a stratification variable for sampling with two
levels, \texttt{"L"} and \texttt{"M"}, which are categorized by
landscape metrics at each site; \texttt{count}, the number of moose at
each site; \texttt{presence}, a factor that indicates whether at least
one moose was observed at each site (\texttt{0} implies no moose;
\texttt{1} implies at least one moose); and \texttt{geometry}, the NAD83
projected coordinate of each site. The \texttt{moose\_preds} data in
\pkg{spmodel} contain spatial locations at which predictions of moose
presence are desired. They are also an \texttt{sf} object with the same
projection and measurements for \texttt{elev} and \texttt{strat}.
Figure~\ref{fig-moose-data} shows the \texttt{presence} variable in
\texttt{moose} as well as the spatial locations of both \texttt{moose}
and \texttt{moose\_preds}. Moose are most common in the southwestern and
eastern parts of the domain and least common in the northwest.

\begin{figure}

\centering{

\includegraphics[width=0.7\textwidth,height=0.7\textheight]{figures/figure-1.png}

}

\caption{\label{fig-moose-data}Moose presence in Alaska. Circles
represent moose presence or absence (based on color) and triangles
represent locations at which moose presence probability predictions are
desired.}

\end{figure}%

To study the effect of elevation, stratum, and their interaction on
moose presence while accounting for spatial autocorrelation, we fit a
SPGLM for binary data (i.e., a spatial logistic regression model) using
\fct{spglm}:

\begin{verbatim}
R> spbin <- spglm(
+  formula = presence ~ elev + strat + elev:strat,
+  family = binomial,
+  data = moose,
+  spcov_type = "spherical"
+)
\end{verbatim}

Summarizing the model object yields a summary similar to that provided
by the familiar \fct{glm}:

\begin{verbatim}
R> summary(spbin)
\end{verbatim}

\begin{verbatim}

Call:
spglm(formula = presence ~ elev + strat + elev:strat, family = binomial,
    data = moose, spcov_type = "spherical")

Deviance Residuals:
    Min      1Q  Median      3Q     Max
-1.8423 -0.7538  0.3883  0.7604  1.6018

Coefficients (fixed):
             Estimate Std. Error z value Pr(>|z|)
(Intercept) -3.039992   1.205695  -2.521  0.01169 *
elev         0.009133   0.004126   2.213  0.02687 *
stratM       3.276511   1.162603   2.818  0.00483 **
elev:stratM -0.010882   0.006697  -1.625  0.10418
---
Signif. codes:  0 '***' 0.001 '**' 0.01 '*' 0.05 '.' 0.1 ' ' 1

Pseudo R-squared: 0.08845

Coefficients (spherical spatial covariance):
       de        ie     range
5.083e+00 2.580e-03 5.158e+04

Coefficients (Dispersion for binomial family):
dispersion
         1
\end{verbatim}

The summary contains the original function call, a summary of residuals,
the fixed effects coefficients table, the spatial covariance parameter
estimates, and additional model information like the pseudo R-squared,
which quantifies the variability in the model attributable to the fixed
effects. While useful, this summary information is hard to work with, as
it is printed directly to the \proglang{R} console. The \pkg{broom}
package from the tidymodels \citep{kuhn2022tidy} ecosystem has functions
to provide helpful model output in the form of tibbles (i.e., data
frames) that are easily manipulated. \pkg{spmodel} has methods for the
\fct{tidy}, \fct{glance}, and \fct{augment} functions from \pkg{broom}.
The first \pkg{broom} function is \fct{tidy}, which tidies the model
output:

\begin{verbatim}
R> tidy(spbin)
\end{verbatim}

\begin{verbatim}
# A tibble: 4 x 5
  term        estimate std.error statistic p.value
  <chr>          <dbl>     <dbl>     <dbl>   <dbl>
1 (Intercept) -3.04      1.21        -2.52 0.0117
2 elev         0.00913   0.00413      2.21 0.0269
3 stratM       3.28      1.16         2.82 0.00483
4 elev:stratM -0.0109    0.00670     -1.62 0.104
\end{verbatim}

The estimates and standard errors returned are on the log odds link
(Table\(~\)\ref{table-links}) scale (\fct{coef} and \fct{vcov} may also
be used). The output provides evidence that elevation is positively
associated with moose presence in the \texttt{"L"} stratum (\(p-\)value
\(<0.05\)) and, at zero elevation, moose are more likely in the
\texttt{"M"} stratum than the \texttt{"L"} stratum (\(p-\)value
\(<0.01\)). This output provides marginal evidence that the effect of
elevation on moose presence varies across strata (\(p-\)value
\(\approx 0.1\)). The model effectively quantifies the impact of
elevation on moose presence for moose in the \texttt{"L"} strata, but an
analogous statement for moose in the \texttt{"M"} strata requires more
context. We could refit the model treating \texttt{"M"} as the reference
group instead of \texttt{"L"}:

\begin{verbatim}
R> moose$strat2 <- factor(moose$strat, levels = c("M", "L"))
R> update(spbin, formula = presence ~ elev + strat2 + elev:strat2) |>
+  summary()
\end{verbatim}

\begin{verbatim}

Call:
spglm(formula = presence ~ elev + strat2 + elev:strat2, family = binomial,
    data = moose, spcov_type = "spherical")

Deviance Residuals:
    Min      1Q  Median      3Q     Max
-1.8423 -0.7538  0.3883  0.7604  1.6018

Coefficients (fixed):
              Estimate Std. Error z value Pr(>|z|)
(Intercept)   0.236519   1.305198   0.181  0.85620
elev         -0.001750   0.006090  -0.287  0.77385
strat2L      -3.276511   1.162603  -2.818  0.00483 **
elev:strat2L  0.010882   0.006697   1.625  0.10418
---
Signif. codes:  0 '***' 0.001 '**' 0.01 '*' 0.05 '.' 0.1 ' ' 1

Pseudo R-squared: 0.08845

Coefficients (spherical spatial covariance):
       de        ie     range
5.083e+00 2.580e-03 5.158e+04

Coefficients (Dispersion for binomial family):
dispersion
         1
\end{verbatim}

A simpler solution, especially if there categorical variables with many
levels, is to leverage \pkg{emmeans}. \pkg{emmeans} is an \proglang{R}
package for estimating marginal means of model objects. The
\fct{emtrends} function in \pkg{emmeans} characterizes the effect of a
continuous variable (here, \texttt{elev}) for each level of a
categorical variable (here, \texttt{strat}):

\begin{verbatim}
R> library("emmeans")
\end{verbatim}

\begin{verbatim}
R> emtrends(spbin, "strat", "elev")
\end{verbatim}

\begin{verbatim}
 strat elev.trend      SE  df asymp.LCL asymp.UCL
 L        0.00913 0.00413 Inf   0.00105    0.0172
 M       -0.00175 0.00609 Inf  -0.01369    0.0102

Degrees-of-freedom method: asymptotic
Confidence level used: 0.95
\end{verbatim}

The asymptotic confidence intervals show that there is more evidence of
an association between elevation and moose presence in the \texttt{"L"}
stratum than in the \texttt{"M"} stratum. Notice that
\texttt{elev.trend} for the \texttt{"L"} stratum matches the
\texttt{elev} effect when \texttt{"L"} is the reference group, and
similarly for \texttt{elev.trend} when \texttt{"M"} is the reference
group.

The second \pkg{broom} function is \fct{glance}, which glances at the
model fit:

\begin{verbatim}
R> glance(spbin)
\end{verbatim}

\begin{verbatim}
# A tibble: 1 x 10
      n     p  npar value   AIC  AICc   BIC logLik deviance pseudo.r.squared
  <int> <dbl> <int> <dbl> <dbl> <dbl> <dbl>  <dbl>    <dbl>            <dbl>
1   218     4     3  681.  687.  687.  697.  -340.     161.           0.0885
\end{verbatim}

\fct{glance} returns several useful statistics like the sample size
(\texttt{n}), number of fixed effects (\texttt{p}), number of covariance
parameters (\texttt{npar}), several likelihood-based statistics (e.g.,
\texttt{AIC}, \texttt{AICc}, \texttt{BIC}), and pseudo R-squared.

\begin{figure}

\centering{

\includegraphics[width=1\textwidth,height=1\textheight]{figures/figure-2.png}

}

\caption{\label{fig-sp-diagnostic}Spatial logistic regression model
diagnostics from \fct{augment}. The leverage (i.e., hat) values (left)
and standardized residuals (right).}

\end{figure}%

The third \pkg{broom} function is \fct{augment}, which augments the
model data with diagnostics:

\begin{verbatim}
R> head(augment(spbin))
\end{verbatim}

\begin{verbatim}
Simple feature collection with 6 features and 8 fields
Geometry type: POINT
Dimension:     XY
Bounding box:  xmin: 281896.4 ymin: 1518398 xmax: 311325.3 ymax: 1541016
Projected CRS: NAD83 / Alaska Albers
# A tibble: 6 x 9
  presence  elev strat .fitted .resid    .hat   .cooksd .std.resid
  <fct>    <dbl> <chr>   <dbl>  <dbl>   <dbl>     <dbl>      <dbl>
1 0         469. L      -1.47  -0.644 0.101   0.0130        -0.679
2 0         362. L      -2.77  -0.349 0.0166  0.000523      -0.352
3 0         173. M      -2.23  -0.451 0.00390 0.000200      -0.452
4 0         280. L      -3.59  -0.234 0.00343 0.0000472     -0.234
5 0         620. L      -0.774 -0.871 0.319   0.130         -1.06
6 0         164. M      -2.01  -0.502 0.00459 0.000292      -0.503
# i 1 more variable: geometry <POINT [m]>
\end{verbatim}

The \fct{augment} function returns the fitted \(\mathbf{w}\) values
(\texttt{.fitted}), deviance residuals (\texttt{.resid}), leverage
(i.e., hat) values (\texttt{.hat}), Cook's distance (\texttt{.cooksd}),
and standardized residuals (\texttt{.std.resid}). When the data are an
\texttt{sf} object, \fct{augment} returns another \texttt{sf} object,
helpful for visualizing model diagnostics spatially as in
Figure~\ref{fig-sp-diagnostic}. Leverage measures the unusualness of an
observation's set of explanatory variables, while Cook's distances
measures how influential an observation is on the resulting model fit
\citep{montgomery2021introduction}. Model diagnostics are also
accessible as vectors using the appropriate generic function (e.g.,
\fct{fitted}, \fct{residuals}).

Similar to \fct{glm} model objects, \fct{plot} can be used to visualize
diagnostics (Figure~\ref{fig-sp-diagnostic2}):

\begin{verbatim}
R> plot(spbin, which = c(4, 7))
\end{verbatim}

\begin{figure}

\centering{

\includegraphics[width=1\textwidth,height=1\textheight]{figures/figure-3.png}

}

\caption{\label{fig-sp-diagnostic2}Spatial logistic regression model
diagnostics from \fct{plot}. The Cook's distance values (left) and the
fitted spatial covariance as a function of distance (right).}

\end{figure}%

Components of model variation are partitioned using \fct{varcomp}:

\begin{verbatim}
R> varcomp(spbin)
\end{verbatim}

\begin{verbatim}
# A tibble: 3 x 2
  varcomp            proportion
  <chr>                   <dbl>
1 Covariates (PR-sq)   0.0885
2 de                   0.911
3 ie                   0.000462
\end{verbatim}

The fixed effects explain roughly 9\% of model variation, while the
spatially dependent variance explains most of the remaining variability.

\begin{figure}

\centering{

\includegraphics[width=0.7\textwidth,height=0.7\textheight]{figures/figure-4.png}

}

\caption{\label{fig-moose-fit}Moose presence probability fitted values
and predictions. Fitted values are represeneted by circles and
predictions by triangles.}

\end{figure}%

We make predictions of the log odds of moose probability presence at
each site in \texttt{moose\_preds} using \fct{predict}:

\begin{verbatim}
R> head(predict(spbin, newdata = moose_preds))
\end{verbatim}

\begin{verbatim}
          1           2           3           4           5           6
 0.08588581 -0.40762380 -1.87510889 -1.14172781  1.45701519 -2.74275553
\end{verbatim}

\fct{augment} may also be used to augment the prediction data with
predictions:

\begin{verbatim}
head(augment(
  spbin,
  newdata = moose_preds,
  type.predict = "response",
  interval = "prediction"
))
\end{verbatim}

\begin{verbatim}
Simple feature collection with 6 features and 5 fields
Geometry type: POINT
Dimension:     XY
Bounding box:  xmin: 291839.8 ymin: 1436192 xmax: 401239.6 ymax: 1512103
Projected CRS: NAD83 / Alaska Albers
# A tibble: 6 x 6
   elev strat .fitted  .lower .upper           geometry
  <dbl> <chr>   <dbl>   <dbl>  <dbl>        <POINT [m]>
1  143. L      0.521  0.0983   0.916 (401239.6 1436192)
2  324. L      0.399  0.0331   0.928 (352640.6 1490695)
3  158. L      0.133  0.00957  0.709 (360954.9 1491590)
4  221. M      0.242  0.0261   0.792 (291839.8 1466091)
5  209. M      0.811  0.289    0.978 (310991.9 1441630)
6  218. L      0.0605 0.00360  0.534 (304473.8 1512103)
\end{verbatim}

Here, we requested predictions on the probability (i.e., response) scale
(Figure~\ref{fig-moose-fit}) alongside lower and upper bounds of a 95\%
(see \texttt{level}) prediction interval (Figure~\ref{fig-moose-int}).

\begin{figure}

\centering{

\includegraphics[width=1\textwidth,height=1\textheight]{figures/figure-5.png}

}

\caption{\label{fig-moose-int}Moose presence probability prediction
intervals. 95\% prediction interval lower bound (left) and 95\%
prediction interval upper bound (right).}

\end{figure}%

Thus far we have heuristically argued, based on first principles, that
there are benefits to incorporating spatial autocorrelation for GLMs
applied to spatial data. Now we provide some empirical evidence to
support this claim by comparing the fits of the SPGLM and a GLM. If
\texttt{spcov\_type\ =\ "none"}, the resulting model fit is nearly
identical to that from \fct{glm}:

\begin{verbatim}
R> # fit nonspatial model using spglm()
R> bin <- spglm(
+  formula = presence ~ elev + strat + elev:strat,
+  family = binomial,
+  data = moose,
+  spcov_type = "none"
+)
R> # fit nonspatial model using glm()
R> bin_glm <- glm(
+  formula = presence ~ elev + strat + elev:strat,
+  family = binomial,
+  data = moose
+)
R> # compare fixed effect coefficients and standard errors up to four decimals
R> data.frame(
+  est_none = coef(bin),
+  est_glm = coef(bin_glm),
+  se_none = sqrt(diag(vcov(bin))),
+  se_glm = sqrt(diag(vcov(bin_glm)))
+) |>
+  apply(2, round, digits = 4)
\end{verbatim}

\begin{verbatim}
            est_none est_glm se_none se_glm
(Intercept)  -0.5219 -0.5219  0.5082 0.5082
elev          0.0002  0.0002  0.0024 0.0024
stratM        1.0274  1.0274  0.7150 0.7150
elev:stratM  -0.0013 -0.0013  0.0038 0.0038
\end{verbatim}

The advantage of using \fct{spglm} to fit a model with
\texttt{spcov\_type\ =\ "none"} is that it provides access to other
\pkg{spmodel} functions for model objects (e.g., \fct{glances} below)
and accounts for the additional terms in the likelihood from
Equation\(~\)\eqref{eq-marginal6}. These additional terms make the
likelihood for \fct{spglm} and \fct{glm} different, though the models
convey the same information. A glance at the spatial and nonspatial
models reveals:

\begin{verbatim}
R> glances(spbin, bin)
\end{verbatim}

\begin{verbatim}
# A tibble: 2 x 11
  model     n     p  npar value   AIC  AICc   BIC logLik deviance
  <chr> <int> <dbl> <int> <dbl> <dbl> <dbl> <dbl>  <dbl>    <dbl>
1 spbin   218     4     3  681.  687.  687.  697.  -340.     161.
2 bin     218     4     0  717.  717.  717.  717.  -359.     294.
# i 1 more variable: pseudo.r.squared <dbl>
\end{verbatim}

The spatial model has a notably lower AIC, AICc, BIC, and deviance,
suggesting it is the superior model. Another model comparison approach
is leave-one-out cross validation. In leave-one-out cross validation,
separately each observation is held out, a model is fit to the remaining
data, and a prediction is made for the mean of the held out observation
on the response scale. Then, statistics like leave-one-out bias,
mean-squared-prediction error (MSPE), and the square root of MSPE
(RMSPE) may be computed:

\begin{verbatim}
R> loocv(spbin) |>
+  apply(2, round, digits = 4)
\end{verbatim}

\begin{verbatim}
   bias    MSPE   RMSPE
-0.0006  0.1458  0.3818
\end{verbatim}

\begin{verbatim}
R> loocv(bin) |>
+  apply(2, round, digits = 4)
\end{verbatim}

\begin{verbatim}
  bias   MSPE  RMSPE
0.0000 0.2403 0.4902
\end{verbatim}

Both models are nearly unbiased, but the spatial model has an
approximately 39\% lower MSPE, suggesting the probability predictions
tend to be much closer to the observed presence values.

A third model comparison tool is area under the receiver operating
characteristic (AUROC) curve. The AUROC curve ranges from zero to one
and conveys a model's classification performance over all possible
probability thresholds \citep{james2013introduction}. Larger values of
AUROC indicate a more accurate model:

\begin{verbatim}
R> AUROC(spbin)
\end{verbatim}

\begin{verbatim}
[1] 0.9490741
\end{verbatim}

\begin{verbatim}
R> AUROC(bin)
\end{verbatim}

\begin{verbatim}
[1] 0.647138
\end{verbatim}

All three performance metrics (likelihood-based statistics,
leave-one-out statistics, and AUROC) prefer the spatial model.

\subsection{Count data}\label{count-data}

\begin{figure}

\centering{

\includegraphics[width=0.7\textwidth,height=0.7\textheight]{figures/figure-6.png}

}

\caption{\label{fig-moose-data-count}Moose counts in Alaska. Circles
represent moose counts (based on color) and triangles represent
locations at which mean count predictions are desired.}

\end{figure}%

The \texttt{count} variable in \texttt{moose} contains the number of
moose observed at a site (Figure~\ref{fig-moose-data-count}). Count data
are often modeled using Poisson or negative binomial regression with the
log link function. The Poisson regression model assumes each datum's
underlying latent mean equals its variance, while the negative binomial
accommodates overdispersion (where the variance is greater than the
mean) at the cost of estimating an extra parameter.

So far our spatial models have made an implicit assumption of geometric
isotropy. A spatial covariance is geometrically isotropic if its
dependence decays with distance equally in all directions. A spatial
covariance is geometrically anisotropic if its dependence decays
differently in different directions. The geometric anisotropy's
directionality and strength are controlled by rotation and scale
parameters that are applied to the original coordinates, creating a
transformed set of coordinates whose spatial covariance is geometrically
isotropic. Geometrically anisotropic models are fit by specifying
\texttt{anisotropy}:

\begin{verbatim}
R> sppois <- spglm(
+  formula = count ~ elev + strat + elev:strat,
+  family = poisson,
+  data = moose,
+  spcov_type = "gaussian",
+  anisotropy = TRUE
+)
R>
R> spnbin <- update(sppois, family = nbinomial)
\end{verbatim}

Because the models have the same support (i.e., both non-negative count
models), we can use likelihood-based statistics to compare them:

\begin{verbatim}
R> glances(sppois, spnbin, sort_by = "AIC") |>
+  subset(select = c(model, npar, AIC, AICc, BIC))
\end{verbatim}

\begin{verbatim}
# A tibble: 2 x 5
  model   npar   AIC  AICc   BIC
  <chr>  <int> <dbl> <dbl> <dbl>
1 spnbin     6 1318. 1319. 1339.
2 sppois     5 1320. 1321. 1337.
\end{verbatim}

The negative binomial model has a slightly lower AIC and AICc, while the
Poisson model has a slightly lower BIC. This is reasonable given the BIC
penalizes additional parameters (here, an overdispersion parameter) more
heavily than AIC and AICc. The leave-out-out MSPE prefers the negative
binomial model:

\begin{verbatim}
R> loocv(sppois) |>
+  apply(2, round, digits = 4)
\end{verbatim}

\begin{verbatim}
   bias    MSPE   RMSPE
 1.2882 31.7882  5.6381
\end{verbatim}

\begin{verbatim}
R> loocv(spnbin) |>
+  apply(2, round, digits = 4)
\end{verbatim}

\begin{verbatim}
   bias    MSPE   RMSPE
 0.3485 28.3760  5.3269
\end{verbatim}

A likelihood-based comparison between the negative binomial anisotropic
model and the negative binomial isotropic model suggests that the
anisotropic model is preferred (lower AIC, AICc, and BIC):

\begin{verbatim}
R> spnbin_iso <- update(spnbin, anisotropy = FALSE)
R> glances(spnbin_iso, spnbin, sort_by = "AIC") |>
+  subset(select = c(model, npar, AIC, AICc, BIC))
\end{verbatim}

\begin{verbatim}
# A tibble: 2 x 5
  model       npar   AIC  AICc   BIC
  <chr>      <int> <dbl> <dbl> <dbl>
1 spnbin         6 1318. 1319. 1339.
2 spnbin_iso     4 1333. 1333. 1346.
\end{verbatim}

\fct{plot} returns the spatial covariance as a function of direction
(Figure~\ref{fig-tropy}):

\begin{verbatim}
R> plot(spnbin_iso, which = 8)
R> plot(spnbin, which = 8)
\end{verbatim}

\begin{figure}

\centering{

\includegraphics[width=1\textwidth,height=1\textheight]{figures/figure-7.png}

}

\caption{\label{fig-tropy}Level curves of equal autocorrelation for the
negative binomial moose count models. The ellipse is centered at zero
distance in the x-direction and y-direction, and points along the
ellipse have equal levels of autocorrelation. In the isotropic level
curve (left), spatial covariance decays equally in all directions. In
the anistropic level curve (right), spatial covariance decays fastest in
the northeast-southwest direction and slowest in the northwest-southeast
direction (this pattern can be seen in the observed counts).}

\end{figure}%

Earlier we used \fct{tidy} to tidy the model's fixed effects, but we can
also use tidy to tidy the spatial covariance parameters:

\begin{verbatim}
R> tidy(spnbin, effects = "spcov")
\end{verbatim}

\begin{verbatim}
# A tibble: 5 x 3
  term    estimate is_known
  <chr>      <dbl> <lgl>
1 de         4.65  FALSE
2 ie         0.148 FALSE
3 range  84013.    FALSE
4 rotate     2.74  FALSE
5 scale      0.259 FALSE
\end{verbatim}

The \texttt{rotate} parameter is the number of radians in \([0, \pi]\)
the ellipse is rotated and the \texttt{scale} parameter is the ratio in
\((0, 1]\) of the minor axis length to the major axis length. The
\texttt{is\_known} column indicates whether the parameter was assumed
known during optimization, controlled by specifying the
\texttt{spcov\_initial} argument.

\subsection{Skewed data}\label{skewed-data}

The \texttt{seal} data in \pkg{spmodel} is an \texttt{sf} object with
data on harbor seal trends in Alaska. The \texttt{log\_trend} variable
is the logarithm of a seal abundance temporal trend measure at the site
(based on historical data), and the \texttt{stock} variable is a factor
with two levels, \texttt{8} and \texttt{10}, where each level represents
one of twelve seal stocks (i.e., breeds) in Alaska. The \texttt{seal}
geometry is an areal polygon geometry and hence, spatial autoregressive
models based on neighborhood distance are appropriate.

SPGLMs for areal data are are fit in \pkg{spmodel} using \fct{spgautor},
which has similar syntax as \fct{spglm} but contains arguments to
control the weight matrix (\(\mathbf{W}\) in
Equation\(~\)\ref{eq-spcov-sar}) and whether or not row-standardization
\citep{ver2018spatial} is applied. By default, polygons are neighbors if
they share a boundary (i.e., Queen's contiguity; see
\citet{pebesma2023spatial}) and row standardization is assumed. Weight
matrices may be provided via the \texttt{W} argument and row
standardization may be ignored via the \texttt{row\_st} arguent.
Following \citet{ver2018spatial}, polygons without a neighbor are given
their own (independent) variance parameter called \texttt{extra}.

The trend data were originally logged to remove skew
\citep{ver2018spatial}, but we will exponentiate \texttt{log\_trend}
(Figure~\ref{fig-seal}) and model this skew directly using a SPGLM:

\begin{verbatim}
R> seal$trend <- exp(seal$log_trend)
\end{verbatim}

The \texttt{trend} variable has several missing (\texttt{NA}) values,
which represent polygons at which predictions of \texttt{trend} are
desired. To make predictions using spatial autoregressive models, the
prediction locations must be known prior to model fitting because these
locations affect the neighborhood structure of the observed data
\citep{ver2018spatial}. This restriction is notably different than
SPGLMs for point-referenced data (i.e., geostatistical models), which
completely separates the estimation and prediction steps.

\begin{figure}

\centering{

\includegraphics[width=1\textwidth,height=1\textheight]{figures/figure-8.png}

}

\caption{\label{fig-seal}Seal trend distribution in Alaska. Observed and
missing seal polygons by stock (left) and observed seal trends (right).}

\end{figure}%

\pkg{spmodel} supports the gamma and inverse Gaussian families for
modeling skewed, positive response variables. \pkg{spmodel} also
supports nonspatial random effects specified via the \texttt{random}
argument, which uses a similar formula syntax as \pkg{nlme}
\citep{pinheiro2006mixed} and \pkg{lme4} \citep{bates2015lme4}. Using
likelihood-based statistics, we compare two models fit using the
simultaneous autoregressive covariance and the Gamma and inverse
Gaussian families. Both models have a random effect for seal stock,
which builds additional correlation into the model for two polygons from
the same stock:

\begin{verbatim}
R> spgam <- spgautor(
+  formula = trend ~ 1,
+  family = Gamma,
+  data = seal,
+  spcov_type = "sar",
+  random = ~ stock
+)
R> spinvg <- update(spgam, family = inverse.gaussian)
R> glances(spgam, spinvg, sort_by = "AIC") |>
+  subset(select = c(model, npar, AIC, AICc, BIC))
\end{verbatim}

\begin{verbatim}
# A tibble: 2 x 5
  model   npar   AIC  AICc   BIC
  <chr>  <int> <dbl> <dbl> <dbl>
1 spinvg     5  108.  109.  121.
2 spgam      5  114.  115.  127.
\end{verbatim}

The inverse Gaussian model has a lower AIC, AICc, and BIC, which
indicates it is a better fit than the gamma model.

We may \fct{tidy} the estimated stock random effect variance:

\begin{verbatim}
R> tidy(spinvg, effects = "randcov")
\end{verbatim}

\begin{verbatim}
# A tibble: 1 x 3
  term      estimate is_known
  <chr>        <dbl> <lgl>
1 1 | stock  0.00411 FALSE
\end{verbatim}

The locations to predict \texttt{trend} (\texttt{NA} values) are stored
in the \texttt{newdata} element of \texttt{spinvg} and used for
prediction:

\begin{verbatim}
R> # output omitted
R> predict(spinvg, type = "response", interval = "prediction")
\end{verbatim}

If using \fct{augment} for prediction, \texttt{newdata} must be
specified:

\begin{verbatim}
R> head(augment(
+  spinvg,
+  newdata = spinvg$newdata,
+  type.predict = "response",
+  interval = "prediction"
+))
\end{verbatim}

\begin{verbatim}
Simple feature collection with 6 features and 6 fields
Geometry type: POLYGON
Dimension:     XY
Bounding box:  xmin: 1030504 ymin: 1012786 xmax: 1115097 ymax: 1057579
Projected CRS: NAD83 / Alaska Albers
# A tibble: 6 x 7
  log_trend stock trend .fitted .lower .upper                           geometry
      <dbl> <fct> <dbl>   <dbl>  <dbl>  <dbl>                      <POLYGON [m]>
1        NA 8        NA   0.942  0.893  0.993 ((1035002 1054710, 1035002 105454~
2        NA 8        NA   0.942  0.893  0.993 ((1043093 1020553, 1043097 102055~
3        NA 8        NA   0.942  0.893  0.993 ((1099737 1054310, 1099752 105426~
4        NA 8        NA   0.942  0.893  0.993 ((1099002 1036542, 1099134 103646~
5        NA 8        NA   0.942  0.893  0.993 ((1076902 1053189, 1076912 105317~
6        NA 8        NA   0.942  0.893  0.993 ((1070501 1046969, 1070317 104659~
\end{verbatim}

\subsection{Proportion data}\label{proportion-data}

We end with two examples of beta regression for proportion data
\citep{ferrari2004beta}. First, we model the nitrogen percentage in a
caribou foraging experiment. Second, we model the proportion of voter
turnout by Texas county in the United States (US) 1980 presidential
election.

\begin{figure}

\centering{

\includegraphics[width=0.7\textwidth,height=0.7\textheight]{figures/figure-9.png}

}

\caption{\label{fig-caribou}Caribou data from a spatial experimental
design measuring the percentage of nitrogen (\texttt{z}) in soil and
testing two factors: tarp type and water presence.}

\end{figure}%

The \texttt{caribou} data in \pkg{spmodel} are a data frame from a
caribou foraging experiment in Alaska meant to study the impact of water
and tarp cover on the percentage of nitrogen in surrounding plants
(Figure~\ref{fig-caribou}). \citet{lenart2002climate} studied these data
treating nitrogen percentage as a continuous variable, but here we treat
nitrogen percentage (\texttt{z}) as a proportion:

\begin{verbatim}
R> spbeta <- spglm(
+    formula = z/100 ~ water + tarp + water:tarp,
+    family = "beta",
+    data = caribou,
+    spcov_type = "matern",
+    xcoord = x,
+    ycoord = y
+)
\end{verbatim}

The nitrogen percentage is dynamically scaled in \texttt{formula} from
(0, 100) to (0, 1) so that it is a proportion. Nitrogen percentage is
modeled as a function of \texttt{water} (two levels: \texttt{"Y"} for
water and \texttt{"N"} for no water), \texttt{tarp} (three levels:
\texttt{"clear"} for a clear tarp, \texttt{"none"} for no tarp, and
\texttt{"shade"} for a shaded tarp), and their interaction, which lets
the effect of water presence vary across tarp type. \texttt{caribou} is
a data frame (not an \texttt{sf} object), so we supply the x-coordinate
and y-coordinate directly via \texttt{xcoord} and \texttt{ycoord}, and,
consistent with the tidyverse approach \citep{wickham2019welcome},
column names in \texttt{data} do not need to be quoted when referenced
(but can be quoted).

A summary of \texttt{spbeta} returns a coefficients table that provides
parameter estimates relative to a reference group. When factors have
more than two levels, it is not straightforward to use these contrasts
to determine overall significance of the factor. The \texttt{anova()}
function tests marginal (i.e., Type III sums of squares) significance of
factors using the general linear hypothesis test for spatial (i.e.,
correlated) data \citep{schabenberger2017statistical}:

\begin{verbatim}
tidy(anova(spbeta))
\end{verbatim}

\begin{verbatim}
# A tibble: 4 x 4
  effects        df statistic       p.value
  <chr>       <int>     <dbl>         <dbl>
1 (Intercept)     1  35783.   0
2 water           1      1.52 0.218
3 tarp            2     38.4  0.00000000468
4 water:tarp      2      5.74 0.0566
\end{verbatim}

These results suggest there is some evidence that the effect of water on
nitrogen percentage differ depending on the type of tarp used (0.01
\textless{} \(p~\)value \textless{} 0.1).

Sometimes averages or contrasts between factor levels that are not in
the reference group are of interest. We again leverage \pkg{spmodel}'s
built-in support for \pkg{emmeans} and use it to obtain the averages of
each factor combination on the link (here, logit) scale:

\begin{verbatim}
spemm <- emmeans(spbeta, ~ water + tarp)
spemm
\end{verbatim}

\begin{verbatim}
 water tarp  emmean     SE  df asymp.LCL asymp.UCL
 N     clear  -3.94 0.0208 Inf     -3.98     -3.90
 Y     clear  -3.90 0.0206 Inf     -3.94     -3.86
 N     none   -3.88 0.0204 Inf     -3.92     -3.84
 Y     none   -3.91 0.0206 Inf     -3.95     -3.87
 N     shade  -3.77 0.0196 Inf     -3.81     -3.73
 Y     shade  -3.83 0.0200 Inf     -3.87     -3.79

Degrees-of-freedom method: asymptotic
Results are given on the logit (not the response) scale.
Confidence level used: 0.95
\end{verbatim}

Delta method \citep{ver2012invented} standard errors are used when
averages on the response (here, proportion) scale are desired:

\begin{verbatim}
update(spemm, type = "response")
\end{verbatim}

\begin{verbatim}
 water tarp  response       SE  df asymp.LCL asymp.UCL
 N     clear   0.0191 0.000390 Inf    0.0183    0.0198
 Y     clear   0.0198 0.000398 Inf    0.0190    0.0205
 N     none    0.0202 0.000404 Inf    0.0194    0.0210
 Y     none    0.0197 0.000398 Inf    0.0189    0.0205
 N     shade   0.0226 0.000434 Inf    0.0218    0.0235
 Y     shade   0.0213 0.000418 Inf    0.0205    0.0221

Degrees-of-freedom method: asymptotic
Confidence level used: 0.95
Intervals are back-transformed from the logit scale
\end{verbatim}

Pairwise contrasts use a Tukey \(p~\)value adjustment
\citep{tukey1949comparing} by default. Here, we request no \(p~\)value
adjustment:

\begin{verbatim}
head(pairs(spemm, adjust = "none"))
\end{verbatim}

\begin{verbatim}
 contrast          estimate     SE  df z.ratio p.value
 N clear - Y clear  -0.0361 0.0293 Inf  -1.233  0.2177
 N clear - N none   -0.0601 0.0292 Inf  -2.063  0.0391
 N clear - Y none   -0.0327 0.0293 Inf  -1.118  0.2637
 N clear - N shade  -0.1735 0.0286 Inf  -6.061  <.0001
 N clear - Y shade  -0.1136 0.0289 Inf  -3.932  0.0001
 Y clear - N none   -0.0241 0.0290 Inf  -0.830  0.4064

Degrees-of-freedom method: asymptotic
Results are given on the log odds ratio (not the response) scale.
\end{verbatim}

\begin{figure}

\centering{

\includegraphics[width=0.7\textwidth,height=0.7\textheight]{figures/figure-10.png}

}

\caption{\label{fig-texas}Proportion of voter turnout in Texas for the
1980 presidential election. Circles represent voter turnout (based on
color) and triangles represent locations at which voter turnout
predictions are desired.}

\end{figure}%

We now model the \texttt{elect80} data in \pkg{spData}
\citep{bivand2024spdata}, which contains voter turnout data by county in
the 1980 US Presidential election \citep{pace1997quick}. The
\texttt{texas} data in \pkg{spmodel} contains a subset of these data in
the state of Texas. These data are point-referenced, but we may still
use autoregressive models if neighborhood distance is determined using
county centroids (i.e., counties whose centroid distance is less than
some cutoff are defined as neighbors). The response variable of
interest, \texttt{turnout}, is the proportion of registered voters in
the county who voted in the election (Figure~\ref{fig-texas}).

\begin{verbatim}
R> spgautor_mods <- spgautor(
+  formula = turnout ~ log_income,
+  family = beta,
+  data = texas,
+  spcov_type = c("car", "sar"),
+  cutoff = 2e5,
+  estmethod = "ml"
+)
\end{verbatim}

We model voter turnout as a function of log income using both the
conditional and simultaneous autoregressive models with a neighbor
distance cutoff of 200 kilometers and the maximum likelihood estimation
method. When a vector is provided to \texttt{spcov\_type} in
\fct{spgautor} (or \fct{spglm}), a model is fit for each spatial
covariance type and stored in a list with name equal to the respective
type. Then it is simple to glance at each model fit:

\begin{verbatim}
R> glances(spgautor_mods) |>
+  subset(select = c(model, npar, AIC, AICc, BIC))
\end{verbatim}

\begin{verbatim}
# A tibble: 2 x 5
  model  npar   AIC  AICc   BIC
  <chr> <int> <dbl> <dbl> <dbl>
1 car       3 -38.3 -38.1 -21.2
2 sar       3 -35.7 -35.5 -18.6
\end{verbatim}

The conditional autoregressive model has the best fit (in terms of AIC,
AICc, and BIC). In this model, there is significant evidence
\texttt{log\_income} is positively related to average voter turnout
(\(p~\)value \textless{} 0.001):

\begin{verbatim}
tidy(spgautor_mods$car)
\end{verbatim}

\begin{verbatim}
# A tibble: 2 x 5
  term        estimate std.error statistic    p.value
  <chr>          <dbl>     <dbl>     <dbl>      <dbl>
1 (Intercept)   -4.74      1.06      -4.47 0.00000783
2 log_income     0.532     0.117      4.55 0.00000531
\end{verbatim}

Another way to assess the impact of \texttt{log\_income} on
\texttt{turnout} is a likelihood ratio test:

\begin{verbatim}
reduced_car <- update(spgautor_mods$car, formula = turnout ~ 1)
tidy(anova(reduced_car, spgautor_mods$car))
\end{verbatim}

\begin{verbatim}
# A tibble: 1 x 5
  full              reduced        df statistic     p.value
  <chr>             <chr>       <dbl>     <dbl>       <dbl>
1 spgautor_mods$car reduced_car     1      24.1 0.000000930
\end{verbatim}

Likelihood ratio tests compare the fit of a ``full'' model compared to a
``reduced'' model that is completely nested within the full model. If
there is evidence the full model explains significantly more information
than the reduced model, the likelihood ratio test \(p~\)value will be
small. Similar to the summary output from the general linear hypothesis
test, the likelihood ratio suggests \texttt{log\_income} is related to
average voter turnout (\(p~\)value \textless{} 0.001).

The default estimation method in \pkg{spmodel} is REML, but note that
these models used maximum likelihood (ML). ML is very similar to REML --
the difference is that for ML, the fixed effects \(\boldsymbol{\beta}\)
are not integrated out of Equation\(~\)\eqref{eq-marginal} but are
rather back-substituted. While REML typically performs better for fixed
effect estimation and prediction \citep{zimmerman2024spatial}, ML allows
likelihood-based comparisons (e.g., AIC) for models with simultaneously
varying fixed effect and covariance structures, while REML
likelihood-based comparisons are only valid for models sharing the same
fixed effect structure (though \citet{gurka2006selecting} provides some
evidence that this restriction may be unnecessary).

The point-referenced \texttt{texas} and \texttt{caribou} data may be
analyzed using \fct{spgautor} or \fct{spglm} and comparisons across
these structures can be made using likelihood-based statistics (as long
as the supports of the response distribution are the same). Put another
way, likelihood-based statistics can be used to determine whether
geostatistical (distance-based) or autoregressive (neighbor-based)
structures perform best when the data are point-referenced.

\section{Discussion}\label{sec-discussion}

SPGLMs are fit in \pkg{spmodel} using a novel application of the Laplace
approximation that marginalizes over the latent (i.e., unobserved) mean,
\(\mathbf{w}\), and the fixed effects, \(\boldsymbol{\beta}\). The
approach is quite flexible and accommodates any general response
distribution and covariance structure. \citet{ver2024marginal} show that
the approach, as implemented in \pkg{spmodel}, generally yields unbiased
estimators with proper interval coverage and often outperforms the
Bayesian approach from \pkg{spBayes}, the INLA approach from
\pkg{R-INLA}, and the automatic differentiation approach from
\fct{glmmTMB}.

\pkg{spmodel}'s \fct{spglm} and \fct{spgautor} functions are similar in
structure and syntax as the base-\proglang{R} \fct{glm} function, easing
the transition from GLMs to SPGLMs. These functions support six response
distributions (Table\(~\)\ref{table-links}) and 20 spatial covariance
functions. \pkg{spmodel} provides several additional features that
accommodate geometric anisotropy, nonspatial random effects, fixing
spatial covariance parameters at known values, data having thousands of
observations (following \citet{ver2023indexing}), incorporating spatial
dependence in machine learning (e.g., random forests;
\citet{breiman2001random}), simulating spatially dependent data (e.g.,
\fct{sprbinom}, \fct{sprpois}), and several others. Learn more at
\url{https://CRAN.R-project.org/package=spmodel} and links therein.

\section*{Computational details}\label{computational-details}
\addcontentsline{toc}{section}{Computational details}

The results in this paper were obtained using \proglang{R} 4.4.0 with
the \pkg{spmodel} 0.9.0 package. Figures were created using the
\pkg{ggplot2} 3.5.1 package \citep{wickham2016ggplot2} and base
\proglang{R}.

\section*{Data and code availability}\label{data-and-code-availability}
\addcontentsline{toc}{section}{Data and code availability}

All writing and code associated with this manuscript is available for
viewing and download on GitHub at
\url{https://github.com/USEPA/spmodel.glm.manuscript}. All data used are
part of the \pkg{spmodel} \proglang{R} package available for download
from CRAN at \url{https://CRAN.R-project.org/package=spmodel}.

\section*{Acknowledgments}\label{acknowledgments}
\addcontentsline{toc}{section}{Acknowledgments}

We would like to thank initial reviewers and editors for feedback that
has greatly improved the manuscript.

The views expressed in this manuscript are those of the authors and do
not necessarily represent the views or policies of the U.S.
Environmental Protection Agency or the National Oceanic and Atmospheric
Administration. Any mention of trade names, products, or services does
not imply an endorsement by the U.S. government, the U.S. Environmental
Protection Agency, or the National Oceanic and Atmospheric
Administration. The U.S. Environmental Protection Agency and the
National Oceanic and Atmospheric Administration do not endorse any
commercial products, services or enterprises.

\section*{References}\label{references}
\addcontentsline{toc}{section}{References}

\renewcommand{\bibsection}{}
\bibliography{references.bib}




\end{document}
